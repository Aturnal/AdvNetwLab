Output of the command \textit{iwconfig wlan1}
\begin{verbatim}
wlan1     IEEE 802.11abgn  ESSID:off/any
          Mode:Managed  Access Point: Not-Associated   Tx-Power=0 dBm
          RTS thr:off   Fragment thr:off
          Encryption key:off
          Power Management:off
\end{verbatim}
\begin{itemize}
\item \textbf{IEEE 802.11agbn}: - This first field shows which wireless MAC technologies the device supports on that interface.\\
%
\item \textbf{ESSID:off/any}	- The second field specifies the network name (if any) of a network consisting of a group of cells connected via repeaters or infrastructure
 where users may roam between the cells. When this field is set to off or any, ESSID checking is disabled.\\
%
\item \textbf{Mode:managed}: Specifies the operating mode of the device and tells us what kind of network it operates in as well as what the function of the node is in the network.\\
I this case the mode is set to Managed which means that the node connects to a network composed of many access points where roaming is enabled.\\
%
\item \textbf{Access Point}: Specifies which access point the card is connected to. When configuring, it is possible to specify which access point the card should try to register to.
An access point can be chosen, it can be set to "any", "auto" or "off".
%
\item \textbf{Tx-power=0dBm}: For network cards supporting multiple transmit powers, this setting allows you to set the transmit power in dBm.
It also allows you to specify power control and radio using "auto", "fixed" and "on" or "off" respectively.\\
%
\item \textbf{RTS thr:off}: This parameter specifies when the RTS/CTS mechanism will be used and can be set to "auto", "fixed" and "off".
 RTS/CTS adds overhead but increases throughput in situaties with hidden nodes or a large number of active nodes.\\
%
\item \textbf{Fragment thr:off}: Similar to the RTS threshold this parameter can be "auto", "fixed" or "off" and allows you to specify the maximum fragment size.
Fragmentation can reduce error penalties in a noisy environment. On some cards, this threshold can be higher than the maximum packet size which will enable Frame Bursting.\\
%
\item \textbf{Encryption key:off}: This parameter allows manipulation of encryption, scrambling keys and security mode. It can be used to change security keys,
disable or enable encryption and change the security mode (implications of this are dependent on the type of card). \\
%
\item \textbf{Power Management:off}: Used to manipulate power management scheme parameters and mode. It allows you to disable or enable power management as well as specify which types
of packets should be received and choose the period between wake ups.\\
\end{itemize}
